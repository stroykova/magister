\documentclass[12pt]{article}

\usepackage[utf8]{inputenc}
\usepackage[english, russian]{babel}

% "А вот теперь --- слайды!"
\usepackage{slides}

% ШРИФТЫ
% Нужны рубленные шрифты -- раскомментируйте стоку ниже.
% Нужны шрифты с засечками --- закомментируйте эту строку.
% \renewcommand{\familydefault}{\sfdefault} % Переключает на рубленный шрифт.
% Шрифты Times и Arial, если стоит пакет cyrtimes.
\IfFileExists{cyrtimes.sty}
    {
        \usepackage{cyrtimespatched}
    }
    {
        % А если Times нету, то будет CM...
    }

% \renewcommand{\seriesdefault}{b} % для шрифта с засечками, это предпочтительно
% \renewcommand{\seriesdefault}{sbc} % для рубленного шрифта


% Прочие пакетики
% Графика.
\usepackage{graphicx}

% Настройка презентации
% Студент и руководитель.
\def\Student{Иванов Иван Иванович}
\def\Advisor{Петров Пётр Петрович}
% \def\Person
% \def\Affilation
% Титул
\def\Title{Ускорение темпов инновационой модернизации презентаций}
% Титул для нижнего колонтитула. Может быть сокращённым
\def\FooterTitle{\Title}
\def\SubTitle{Квалификационная работа}
% Title Page

%\maketitle

\thispagestyle{empty}

\begin{center}
\begin{small}
Московский государственный технический университет им. Н.Э. Баумана
\end{small}

\vspace{6cm}

\begin{Large}\textbf{Курсовая работа}\end{Large}\\

по курсу \\
<<Протоколы вычислительных сетей>> \\
\begin{Large}
Создание SMTP-сервера, обеспечивающего локальную доставку и добавление в очередь удаленной доставки.
\end{Large}

\vspace{4cm}

\begin{flushright}
студент	Стройкова Ксения Александровна\\
группа ИУ7--39\\

\end{flushright}

\vspace{8cm}

\begin{small}Москва, 2014\end{small}

\end{center}


% Верхний заголовок: пустой
% Нижний заголовок по-умолчанию:
% \lfoot{\Title} % слева
% \cfoot{} % цент пуст
% \rfoot{\thepage} % справа

% \renewcommand{\baselinestretch}{1.5}
% \linespread{1.6}


%% Переносы в презентации смотряся не очень.
\hyphenpenalty 10000
\sloppy


\begin{document}

% \raggedright -- грубое выавнивание по левому краю.
% \raggedright

\TitleSlide

% Команды section и subsection начинают новый слайд.

\section{Цель и задачи работы}

\emph{Целью работы} является создание шаблона презентаций академического стиля.

\subsubsection{Решаемые задачи}

\begin{enumerate}
\item Выбрать из существующих стилей презентаций единственный стиль, наиболее близкий к желаемому.
\item Доработать его в соответствии с собственными требованиями.
\item Создать тестовые примеры, демонстрирующие возможности стиля.
\item Проверить текстовые примеры в разных дистрибутивах Latex.
\end{enumerate}

\section{Сравнение списков}

Простой текст без списков.

\begin{itemize}
\item Список без номеров (\verb+itemize+).
\item Длинный-предлинный элемент списка с переносом на следующую строку.
\end{itemize}

Простой текст без списков.

Второй абзац простого текста.

\begin{enumerate}
\item Пронумерованный список (\verb+enumerate+).
\item Длинный-предлинный элемент списка с переносом на следующую строку.
\end{enumerate}

\section{Создание слайда}

\begin{enumerate}
\item Добавление нового слайда: \verb+\section{Заголовок}+.
\item Можно использовать множество команд:
\begin{itemize}
\item нумерованные и ненумерованые списки;
\item формул ($e=mc^2$);
\item моноширный текст (\verb+\section+);
\item разные размеры шрифтов: \scriptsize scriptsize, \tiny tiny, \small small,
\normalsize normalsize, \large large, \Large Large, \LARGE LARGE \normalsize\dots;
\item таблицы и фигуры.
\end{itemize}
\item Latex регулирует интервалы между абзацами и перечислениями для улучшения вида страницы.
\end{enumerate}

\section{Вёрстка слайда в две колонки}

\begin{minipage}[m]{.34\textwidth}
\includegraphics[width=\textwidth]{sample-image}
\centerline{Результаты}
\end{minipage}
%
\begin{minipage}[m]{.65\textwidth}
\raggedright
Для вёрстки отдельных слайдов в две колонки используется окружение
  \verb+minipage+. Здесь использован \verb+\raggedright+ для временного выравнвиания влево.

Пример кода с неравномерными колонками (для равномерных используйте 49 и 49):

\begin{verbatim}
\begin{minipage}[m]{.34\textwidth}
  \includegraphics...
\end{minipage}
\begin{minipage}[m]{.64\textwidth}
  Текст
\end{minipage}
\end{verbatim}

% Недостаток использования \verb+minipage+ --- исчезают интервалы между абзацами.
\end{minipage}

\section{Выводы}

\begin{enumerate}
\item Шаблон презентации в целом отвечает поставленным требованиям.

\item У шаблона в настоящий момент имеются следующие недостатки:
\begin{itemize}
\item при использовании \verb+minipage+ исчезает межабзацный интервал;
\item формулы по-умолчанию несколько меньше текста (при использовании \texttt{cyrtimes});
\item используется выравнивание <<по ширине>>.
\end{itemize}

\end{enumerate}


\end{document}

