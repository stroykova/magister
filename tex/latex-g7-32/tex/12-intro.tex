\Introduction

Развитие интернета, рост производительности компьютеров и прогресс в технологии 
производства цифровых камер, сканеров и принтеров привели к широкому использованию 
цифровых данных, в том числе видео. В последнее время наблюдается бурное развитие 
телекоммуникационных систем, предназначенных для приема и передачи 
видеоданных. 
Для хранения видео информации требуется 
больший объем, чем для других типов данных, таких как звуковая, текстовая информация или изображения. 
С ростом разрешающей способности экранов современных персональных компьютеров, а так же экранов кинотеатров
качество видео и их размер постоянно растут. 
Размер графических данных файла с видео пропорционален количеству кадров, количеству пикселей в 
каждом кадре и количеству битов, требуемых для представления глубины цвета каждого пикселя. 
Без использования алгоритмов сжатия файл с графическими данными может занимать объем, сопоставимый с 
емкостью носителей современных персональных компьютеров. 
Необходимо улучшать алгоритмы сжатия данных, представляющих цифровой поток видеоданных. 
Сжатие данных важно как для скорости передачи, так и для эффективности хранения \cite{Pup03}.

В настоящее время разработаны алгоритмы сжатия без потерь на основе универсальных 
методов сжатия и алгоритмы сжатия с потерями, использующие особенности графических 
данных. Продолжаются работы над алгоритмами сжатия с потерями, сохраняющими качество 
видео на высоком уровне.

Области применения методов кодирования и сжатия видеоинформации весьма разнообразны:
от передачи и хранения видео до спутниковых цифровых телекоммуникационных систем. 
Внимание к сжатию видеоинформации особенно возросло в последнее десятилетие в связи с 
разработкой принципиально новых цифровых телекоммуникационных систем. Создание новейших
цифровых устройств обработки, передачи и хранения видеоизображений связано с радикальным
изменением технологических возможностей новейших процессорных систем, создаваемых ведущими
мировыми фирмами, специализирующимися в области совершенствования аппаратных и программных 
компьютерных средств. Использование новейших процессоров с производительностью несколько 
миллиардов операций в секунду обеспечивает реализацию самых сложных и вычислительно емких 
алгоритмов сжатия, что невозможно было осуществить ранее \cite{Pup07}. 

Передача цифрового видео от источника (видеокамера или записанный 
видеоролик) к получателю (видеодисплей) вовлекает в разработку целую цепь 
различных компонентов и процессов. Ключевыми звеньями этой цепи являются 
процесс компрессии (кодирования) и декомпрессии (декодирования), при 
которых несжатый цифровой видеосигнал сокращается до размеров, 
подходящих для его передачи и хранения, а затем восстанавливается для 
отображения на видеоэкране. 
Продуманная разработка процессов компрессии и декомпрессии может 
дать существенное коммерческое и техническое преимущество продукта, 
обеспечив лучшее качество видеоизображения, большую надежность и гибкую 
приспособляемость по сравнению с конкурирующими решениями. 
Таким образом, имеется заинтересованность в развитии и 
улучшении методов компрессии и декомпрессии видео \cite{Pup13}.