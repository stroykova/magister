\Defines % Необходимые определения. Вряд ли понадобться
\begin{description}
\item[Вейвлетное преобразование (англ. wavelet transform)] Инструмент, разбивающий данные, или функции, 
или операторы на составляющие с разными частотами, каждая из которых затем изучается с разрешением, подходящим по масштабу \cite{Pup01}
\end{description}
\begin{description}
\item[Битрейт] Величина потока данных, передаваемого в реальном времени (минимальный размер канала, который сможет пропустить этот поток без задержек). Частный случай — битрейт сжатого звука или видео.
\end{description}
\begin{description}
\item[Класс изображений/видео] совокупность изображений/видео, применение к которым алгоритма архивации дает качественно одинаковые результаты \cite{Pup02}
\end{description}
\begin{description}
\item[Коэффициент сжатия] отношение длины сжатых данных к длине соответствующих им несжатых данных \cite{Pup02}
\end{description}
\begin{description}
\item[Степень сжатия] отношение длины несжатых данных к длине соответствующих им сжатых данных \cite{Pup02}
\end{description}
\begin{description}
\item[Порядковая статистика (order statistic)] i-я порядковая статистика множества, состоящего из n элементов – это i-й элемент в порядке возрастания \cite{Pup05}
\end{description}
\begin{description}
\item[Цифровое видео] последовательность кадров, в которой каждый кадр рассматривается как набор отсчетов аналогового изображения. Отсчеты - пиксели \cite{Pup07}
\end{description}
%%% Local Variables:
%%% mode: latex
%%% TeX-master: "rpz"
%%% End:
