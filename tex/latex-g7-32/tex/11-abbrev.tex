\Abbreviations %% Список обозначений и сокращений в тексте
\begin{description}
\item[CWT] continuous wavelet transform, непрерывное вейвлетное преобразование
\end{description}
\begin{description}
\item[DWT] discrete wavelet transform, дискретное вейвлетное преобразование
\end{description}
%%% Local Variables:
%%% mode: latex
%%% TeX-master: "rpz"
%%% End:
\begin{description}
\item[YUV] цветовая модель, в которой цвет представляется как 3 компоненты — яркость (Y) и две цветоразностных (U и V)
\end{description}

\begin{description}
\item[YDbDr] цветовая модель, в которой цвет представляется как 3 компоненты — яркость (Y) и две цветоразностных (Db и Dr)
\end{description}

\begin{description}
\item[YIQ] цветовая модель, в которой цвет представляется как 3 компоненты — яркость (Y) и две цветоразностных (I и Q)
\end{description}

\begin{description}
\item[RGB (ARGB)] аддитивная цветовая модель, в которой цвет представляется как 3 компоненты — красная (R), синяя (B) и зеленая (G), 
значение которых добавляется к значению черного цвета. Возможно наличие компоненты A - альфа канала
\end{description}

\begin{description}
\item[HSV] цветовая модель, в которой цвет представляется как 3 компоненты - Hue — цветовой тон, 
Saturation — насыщенность, Value (значение цвета) или Brightness — яркость
\end{description}

\begin{description}
\item[CMYK] цветовая модель, в которой цвет представляется компонентами: C - Cyan, M - Magenta, Y - Yellow, B - Black
\end{description}

\begin{description}
\item[PSNR] пиковое отношение сигнала к шуму - соотношение между максимумом возможного значения сигнала и мощностью шума, искажающего значения сигнала
\end{description}

\begin{description}
\item[SSIM] индекс структурного сходства (от англ. structure similarity) - один из методов измерения схожести между двумя изображениями
\end{description}


