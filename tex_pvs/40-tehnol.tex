\chapter{Технологический раздел}

Раздел содержит описание решения технологических задач.

\section{Описания файла конфигурации, параметров командной строки}
Рассмотрим минимальный список поддерживаемых параметров командной строки при запуске SMTP-сервера:

\begin{description}

\item[MailDirPath] корневой каталог для сообщений в формате Maildir;
\item[LogFileName] имя файла журнала (лог);
\item[WorkersCount] число рабочих процессов;
\item[Port] порт привязки сервера SMTP;
\item[DefaultFolder] имя папки для хранения почты для удаленной доставки;
\end{description}

\section{Требования к системе}

Ниже указаны версии основных использованных утилит и библиотек:
\begin{itemize}
\item GNU C Compiler 4.5.3.
\item GNU Make 3.81.
\item cfsm 
\end{itemize}

\section{Тестирование программы}
Далее приведены описание и результаты  системного тестирования.

\subsection{Результаты системного тестирования}

\subsubsection{Успешная отправка письма}

\begin{verbatim}
receive: 
rc = 20
sock = 3
220 Service ready
Send: 
HELO
receive: 
250 OK
Send: 
MAIL FROM: rcpt@domain.name
250 OK
Send: 
RCPT TO: rcpt@domain.name
250 OK
Send: 
DATA
250 OK
Send: 
123
250 OK
Send: 
123
250 OK
Send: 
.
250 OK
Send: 
QUIT
250 OK
\end{verbatim}



